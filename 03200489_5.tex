%\documentclass[a4paper]{jarticle} 
\documentclass[dvipdfmx,a4paper]{jsarticle}
\usepackage{tikz}
\usepackage{amsmath}
\usepackage{amssymb}
\topmargin = 0mm
\oddsidemargin = 5mm
\textwidth = 152mm
\textheight = 240mm


% サブセクションを 問1,問2 にする設定
\renewcommand{\thesection}{[\arabic{section}]}

% サブサブセクションを (1),(2)にする設定
\renewcommand{\thesubsection}{(\arabic{subsection})}
% (i),(ii)なら \arabic を \roman に変える。    (a),(b)なら \alph

\renewcommand{\thesubsubsection}{(\roman{subsubsection})}

% 大問2の3番目の計算式のラベルを (2.3) にする設定
% 計算式の参照には \eqref{eq:hoge} を使う
\makeatletter
  \renewcommand{\theequation}{\arabic{subsection}.\arabic{equation}}
  \@addtoreset{equation}{subsection}
\makeatother

% --------------------------------------------------------------------
\begin{document}

% タイトル
\begin{center}
\textbf{\huge{数学2D演習 第5回}}
\end{center}

%名前
\begin{flushright}
工学部電気電子工学科3年 03200489 末吉七海\\
\end{flushright}

% --------------------------------------------------------------------
% 問1
\section{復習}

%(1)
\subsection{$\log{i}$}
nを整数として、
\begin{align*}
\log{i} &= \log{\Bigl(\exp{\bigl(i(\frac{\pi}{2} + 2n\pi )\bigr)}\Bigr)}\\
&= i(\frac{\pi}{2} + 2n\pi) \quad (n \in \mathbb{Z})
\end{align*}
\\
%(2)
\subsection{$\mbox{\large$i^\frac{1}{2}$}$}
nを整数として、
\begin{align*}
i^\frac{1}{2} &= \Bigl(\exp{\bigl(i(\frac{\pi}{2} + 2n\pi )\bigr)}\Bigr)^{\frac{1}{2}}\\
&= \exp{\bigl(\frac{1}{2}i(\frac{\pi}{2} + 2n\pi )\bigr)}\\
&= \exp{\bigl(i(\frac{\pi}{4} + n\pi )\bigr)}\\
&= (-1)^{n-1}\frac{1 + i}{\sqrt{2}} \quad (n \in \mathbb{Z})
\end{align*}
\\
%(3)
\subsection{$\mbox{\large$i^i$}$}
nを整数として、
\begin{align*}
i^i &= \Bigl(\exp{\bigl(i(\frac{\pi}{2} + 2n\pi )\bigr)}\Bigr)^i\\
&= \exp{\bigl(-(\frac{\pi}{2} + 2n\pi )\bigr)} \quad (n \in \mathbb{Z})
\end{align*}
\\
%(4)
\subsection{$\sin{(i)}$}
\begin{align*}
\sin{(i)} &= \frac{1}{2i}\bigl(e^{ii} - e^{-ii}\bigr)\\
&= \frac{1}{2i}\bigl(e^{-1} - e^{1}\bigr)\\
&= \frac{i}{2}\bigl(e - \frac{1}{e}\bigr)
\end{align*}
\\
%(5)
\subsection{$\log{(1 + i\sqrt{3})}$}
nを整数として、
\begin{align*}
\log{(1 + i\sqrt{3})} &= \log{\Bigl(2\exp{\bigl(i(\frac{\pi}{3} + 2n\pi )\bigr)}\Bigr)}\\
&= \log{2} + i(\frac{\pi}{3} + 2n\pi) \quad (n \in \mathbb{Z})
\end{align*}
\\
%(6)
\subsection{$\mbox{\Large$\frac{2+i}{3-2i}$}$}
\begin{align*}
\frac{2+i}{3-2i} &= \frac{(2+i)(3+2i)}{(3-2i)(3+2i)}\\
&= \frac{4 + 7i}{13}
\end{align*}
\\
%(7)
\subsection{$\tan{\bigl(i + \mbox{\large$\frac{\pi}{3}$}\bigr)}$}
\begin{align*}
\sin{(i)} &= \frac{1}{2i}\bigl(e^{ii} - e^{-ii}\bigr)\\
&= \frac{1}{2i}\bigl(e^{-1} - e^{1}\bigr)\\
&= \frac{i}{2}\bigl(e - \frac{1}{e}\bigr)\\
\cos{(i)} &= \frac{1}{2}\bigl(e^{ii} + e^{-ii}\bigr)\\
&= \frac{1}{2}\bigl(e^{-1} + e^{1}\bigr)\\
&= \frac{1}{2}\bigl(e + \frac{1}{e}\bigr)\\
\tan{(i)} &= \frac{\sin{(i)}}{\cos{(i)}}\\
&= i \frac{e^2-1}{e^2+1}\\
\tan^2{(i)}&= -\frac{(e^2-1)^2}{(e^2+1)^2}\\
\end{align*}
\begin{align*}
\tan{(i + \frac{\pi}{3})} &= \frac{\tan{i}+\tan{\frac{\pi}{3}}}{1 - \tan{i}\tan{\frac{\pi}{3}}}\\
&= \frac{\tan{i}+\sqrt{3}}{1 - \sqrt{3}\tan{i}}\\
&= \frac{\sqrt{3}(1 + \tan^2{i}) + 4\tan{i}}{1 + 3\tan^2{i}}\\
&= \frac{\sqrt{3}(1 -\frac{(e^2-1)^2}{(e^2+1)^2}) + 4i \frac{e^2-1}{e^2+1}}{1 - 3\frac{(e^2-1)^2}{(e^2+1)^2}}\\
&= \frac{\sqrt{3}\bigl((e^2+1)^2 - (e^2-1)^2\bigr) + 4i (e^4-1)}{(e^2+1)^2 - 3(e^2-1)^2}\\
&= \frac{4\sqrt{3}e^2 + 4i (e^4-1)}{4e^4 - 4e^2 + 4}\\
&= \frac{\sqrt{3}e^2 + i (e^4-1)}{e^4 - e^2 + 1}
\end{align*}
\\\\

%[2]
\section{Laurent展開(1)}
$z = e^{i\theta}$とおくと、
\begin{align*}
J_n(w) &= \frac{1}{2\pi i}\oint_{C}\frac{f(z,w)}{z^{n+1}}\mathrm{d}z\\
&= \frac{1}{2\pi i}\int_{-\pi}^{\pi}\frac{\exp{\bigl(\frac{w}{2}(e^{i\theta}-e^{-i\theta})\bigr)}}{\exp{\bigl(i\theta(n+1)\bigr)}}ie^{i\theta}\mathrm{d}\theta\\
&=\frac{1}{2\pi}\int_{-\pi}^{\pi}\frac{\exp{\bigl(wi\sin{\theta}\bigr)}}{\exp{\bigl(i\theta n\bigr)}}\mathrm{d}\theta\\
&= \frac{1}{2\pi}\int_{-\pi}^{\pi}\exp{\bigl(i(w\sin{\theta} - n\theta)\bigr)}\mathrm{d}\theta\\
&= \frac{1}{2\pi}\int_{-\pi}^{\pi}\cos{(w\sin{\theta} - n\theta)} + i\sin{(w\sin{\theta} - n\theta)}\mathrm{d}\theta\\
&= \frac{1}{2\pi}\int_{-\pi}^{\pi}\cos{(w\sin{\theta} - n\theta)}\mathrm{d}\theta \quad (\because \sin{(w\sin{\theta} - n\theta)}は奇関数)
\end{align*}
以上より、示された。
\\\\

%[3]
\section{Laurent展開(2)}
%(1)
\subsection{}
$$
\frac{1}{3z^2 - 5z -2} = \frac{1}{7}\biggl(\frac{1}{z-2} - \frac{3}{3z+1}\biggr)
$$

%(2)
\subsection{}
\subsubsection{$|z| < \mbox{\large$\frac{1}{3}$}$}
\begin{align*}
\frac{1}{3z^2 - 5z -2} &= \frac{1}{7}\frac{1}{z-2} - \frac{3}{7}\frac{1}{3z+1}\\
&=-\frac{1}{14}\frac{1}{1 - \frac{z}{2}} - \frac{3}{7}\frac{1}{1 + 3z}\\
&=\frac{1}{14}\sum_{n = 0}^{\infty}\bigl(\frac{z}{2}\bigr)^n - \frac{3}{7}\sum_{n = 0}^{\infty}(-3z)^n
\end{align*}

\subsubsection{$\mbox{\large$\frac{1}{3}$} < |z| < 2$}
\begin{align*}
\frac{1}{3z^2 - 5z -2} &= \frac{1}{7}\frac{1}{z-2} - \frac{3}{7}\frac{1}{3z+1}\\
&= -\frac{1}{14}\frac{1}{1 - \frac{z}{2}} - \frac{3}{7}\frac{1}{3z}\frac{1}{1 + \frac{1}{3z}}\\
&=-\frac{1}{14}\sum_{n = 0}^{\infty}(\frac{z}{2})^n + \frac{3}{7}\sum_{n = 0}^{\infty}(-\frac{1}{3z})^{n+1}
\end{align*}

\subsubsection{$|z| > 2$}
\begin{align*}
\frac{1}{3z^2 - 5z -2} &= \frac{1}{7}\frac{1}{z-2} - \frac{3}{7}\frac{1}{3z+1}\\
&= \frac{1}{7}\frac{1}{z}\frac{1}{1 - \frac{2}{z}} - \frac{3}{7}\frac{1}{3z}\frac{1}{1 + \frac{1}{3z}}\\
&=\frac{1}{14}\sum_{n = 0}^{\infty}(\frac{2}{z})^{n+1} + \frac{3}{7}\sum_{n = 0}^{\infty}(-\frac{1}{3z})^{n+1}
\end{align*}
\\

\subsection{}
\begin{align*}
\oint_{|z| = 1}z^m\mathrm{d}z = \begin{cases}
2\pi i\quad(m=-1)\\
0\quad(m=0)
\end{cases}
\end{align*}
なので、(2)における$\mbox{\large$\frac{1}{3}$} < |z| < 2$におけるLairent展開を項別積分すると、
\begin{align*}
\oint_{|z| =1}\frac{\mathrm{d}z}{3z^2-5z^2} &= -\frac{1}{14}\sum_{n = 0}^{\infty}\oint_{|z| = 1}(\frac{z}{2})^n\mathrm{d}z  + \frac{3}{7}\sum_{n = 0}^{\infty}\oint_{|z| = 1}(-\frac{1}{3z})^{n+1}\mathrm{d}z \\
&=  \frac{3}{7}\oint_{|z| = 1}-\frac{1}{3z}\mathrm{d}z\\
&= -\frac{1}{7}\oint_{|z| = 1}z^{-1}\mathrm{d}z\\
&= -\frac{2\pi}{7}i
\end{align*}


%[4]
\section{$\mbox{\Large$\int^{2\pi}_0$}R(\cos{\theta},\sin{\theta})\mathrm{d}\theta$型の積分}
$z = e^{i\theta}$と置換すると、積分範囲は$|z|=1$、$\mathrm{d}z = \mbox{\Large$\frac{1}{iz}$}\mathrm{d}\theta$
%(1)
\subsection{}
$z = e^{i\theta}$と置換すると、
\begin{align*}
\int_0^{2\pi}\frac{\mathrm{d}\theta}{a+\cos{\theta}} &= \oint_{|z|=1}\frac{\mathrm{d}z}{iz(a+\frac{z + z^{-1}}{2})}\\
&= \frac{2}{i}\oint_{|z|=1}\frac{\mathrm{d}z}{z^2 +2az + 1}
\end{align*}
$a>1$より、積分経路の内側にある極は、$z = - a + \sqrt{a^2 -1}$\\
$z = - a + \sqrt{a^2 -1}$における留数は、
\begin{align*}
Res\Bigl(\frac{1}{z^2 +2az + 1}, z = - a + \sqrt{a^2 -1}\Bigr) &= \lim_{z \to - a + \sqrt{a^2 -1}}\bigl(z - (- a + \sqrt{a^2 -1})\bigr)\frac{1}{z^2 +2az + 1}\\
&= \lim_{z \to - a + \sqrt{a^2 -1}}\frac{1}{z - (- a - \sqrt{a^2 -1})}\\
&=\frac{1}{- a + \sqrt{a^2 -1} - (- a - \sqrt{a^2 -1})}\\
&= \frac{1}{2\sqrt{a^2 -1}}
\end{align*}
よって、
\begin{align*}
\int_0^{2\pi}\frac{\mathrm{d}\theta}{a+\cos{\theta}} &= \frac{2}{i}\oint_{|z|=1}\frac{\mathrm{d}z}{z^2 +2az + 1}\\
&= \frac{2}{i}\times 2\pi i\times\frac{1}{2\sqrt{a^2 -1}}\\
&= \frac{2\pi}{\sqrt{a^2 -1}}
\end{align*}
\\

%(2)
\subsection{}
$z = e^{i\theta}$と置換すると、
\begin{align*}
\int_0^{2\pi}\mathrm{d}\theta \frac{\cos{2\theta}}{1-2a\cos{\theta}+a^2} &= \oint_{|z|=1}\mathrm{d}z\frac{\frac{z^2+z^{-2}}{2}}{iz(1-2a\frac{z + z^{-1}}{2}+a^2)}\\
&= -\frac{1}{2ai}\oint_{|z|=1}\mathrm{d}z\frac{z^4+1}{z^2(z^2 - (a + \frac{1}{a})z +1)}\\
&= -\frac{1}{2ai}\oint_{|z|=1}\mathrm{d}z\frac{z^4+1}{z^2(z-a)(z-\frac{1}{a})}
\end{align*}
$0<a<1$より、積分経路の内側にある極は、$z = 0,a$\\
2位の極$z = 0$における留数は、
\begin{align*}
Res\Bigl(\frac{z^4+1}{z^2(z-a)(z-\frac{1}{a})}, z = 0\Bigr) &= \lim_{z \to 0}\frac{\mathrm{d}}{\mathrm{d}z}\Bigl(z^2\frac{z^4+1}{z^2(z-a)(z-\frac{1}{a})}\Bigr)\\
&= \lim_{z \to 0}\frac{\mathrm{d}}{\mathrm{d}z}\Bigl(\frac{z^4+1}{z^2 - (a + \frac{1}{a})z +1}\Bigr)\\
&= \lim_{z \to 0}\frac{z^3(z^2 - (a + \frac{1}{a})z +1\bigr) - (z^4+1)(2z - a - \frac{1}{a})}{\bigl(z^2 - (a + \frac{1}{a})z +1\bigr)^2}\\
&= a + \frac{1}{a}
\end{align*}
1位の極$z = a$における留数は、
\begin{align*}
Res\Bigl(\frac{z^4+1}{z^2(z-a)(z-\frac{1}{a})}, z = a\Bigr) &= \lim_{z \to a}(z-a)\frac{z^4+1}{z^2(z-a)(z-\frac{1}{a})}\\
&= \lim_{z \to a}\frac{z^4+1}{z^2(z-\frac{1}{a})}\\
&= \frac{a^4+1}{a(a^2-1)}
\end{align*}
よって、
\begin{align*}
\int_0^{2\pi}\mathrm{d}\theta \frac{\cos{2\theta}}{1-2a\cos{\theta}+a^2} &= -\frac{1}{2ai}\oint_{|z|=1}\mathrm{d}z\frac{z^4+1}{z^2(z-a)(z-\frac{1}{a})}s\frac{1}{2\sqrt{a^2 -1}}\\
&=-\frac{1}{2ai}\Bigl(2\pi i(a + \frac{1}{a}) +2\pi i (\frac{a^4+1}{a(a^2-1)})\Bigr)\\
&=-\frac{\pi}{a}\Bigl(\frac{a^4-1}{a(a^2-1)} + \frac{a^4+1}{a(a^2-1)}\Bigr)\\
&=-\frac{2\pi a^2}{a^2-1}
\end{align*}
\\\\


%[5]
\section{$\mbox{\Large$\int$}^{\infty}_{-\infty}R(x)\mathrm{d}x$型の積分}
%(1)
\subsection{}
$$
I = \int^{\infty}_{-\infty}\frac{\mathrm{d}x}{x^2 - 2x + 2}
$$
とおく。
$$
\int_{C_x}\frac{\mathrm{d}z}{z^2 - 2z + 2} = \int^{R}_{-R}\frac{\mathrm{d}x}{x^2 - 2x + 2} \to I \quad(R \to \infty)
$$
$z = Re^{i\theta}$とおくと、
\begin{align*}
\biggl|\int_{C_R}\frac{\mathrm{d}z}{z^2 - 2z + 2}\biggr| &= \biggl|\int^{\pi}_{0}\frac{i Re^{i\theta}\mathrm{d}\theta}{R^2e^{2i\theta} - 2Re^{i\theta} + 2}\biggr|\\
&= \int^{\pi}_{0}\Bigl|\frac{i Re^{i\theta}\mathrm{d}\theta}{R^2e^{2i\theta} - 2Re^{i\theta} + 2}\Bigr| \to 0 \quad(R \to \infty)\\
\therefore \quad\int_{C_R}\frac{\mathrm{d}z}{z^2 - 2z + 2} &\to 0 \quad(R \to \infty)
\end{align*}
経路$C_x + C_R$の内側にある$\mbox{\Large$\frac{1}{z^2 - 2z + 2}$}$の極は$z = 1+i$であり、一位の極$z = 1+i$における留数は、
\begin{align*}
Res\Bigl(\frac{1}{z^2 - 2z + 2}, z = 1+i\Bigr) &= \lim_{z \to 1+i}\bigl(z - (1+i)\bigr)\frac{1}{z^2 - 2z + 2}\\
&= \lim_{z \to 1+i}\frac{1}{z - (1-i)}\\
&= \frac{1}{1+i - (1-i)}\\
&= \frac{1}{2i}\\
&= -\frac{1}{2}i
\end{align*}
より、
\begin{equation}
\label{3}
\int_{C_x + C_R}\frac{\mathrm{d}z}{z^2 - 2z + 2} = 2\pi i \times \bigl(-\frac{1}{2}i\bigr) = \pi
\end{equation}
一方、
\begin{equation}
\label{4}
\int_{C_x + C_R}\frac{\mathrm{d}z}{z^2 - 2z + 2} = \int_{C_x}\frac{\mathrm{d}z}{z^2 - 2z + 2} + \int_{C_R}\frac{\mathrm{d}z}{z^2 - 2z + 2} = I
\end{equation}
(\ref{3})、(\ref{4})より、
$$
\therefore I = \pi
$$
\\

%(2)
\subsection{}
$$
I = \int^{\infty}_{-\infty}\mathrm{d}x\frac{x^2 - x + 2}{x^4 + 10x^2 + 9}
$$
とおく。
$$
\int_{C_x}\mathrm{d}z\frac{z^2 - z + 2}{z^4 + 10z^2 + 9} = \int^{R}_{-R}\mathrm{d}x\frac{x^2 - x + 2}{x^4 + 10x^2 + 9} \to I \quad(R \to \infty)
$$
$z = Re^{i\theta}$とおくと、
\begin{align*}
\biggl|\int_{C_R}\mathrm{d}z\frac{z^2 - z + 2}{z^4 + 10z^2 + 9}\biggr| &= \biggl|\int^{\pi}_{0}i Re^{i\theta}\mathrm{d}\theta\frac{R^2e^{2i\theta} - Re^{i\theta} + 4}{R^4e^{4i\theta} + 10R^2e^{2i\theta} + 9}\biggr|\\
&= \int^{\pi}_{0}\mathrm{d}\theta\Bigl|\frac{R^3e^{3i\theta} - R^2e^{2i\theta} + 4Re^{i\theta}}{R^4e^{4i\theta} + 10R^2e^{2i\theta} + 9}\Bigr| \to 0 \quad(R \to \infty)\\
\therefore \quad\int_{C_R}\mathrm{d}z\frac{z^2 - z + 2}{z^4 - 10z^2 + 9} &\to 0 \quad(R \to \infty)
\end{align*}
ここで、
$$z^4 + 10z^2 + 9 = (z+i)(z-i)(z+3i)(z-3i)$$
より、
経路$C_x + C_R$の内側にある$\mbox{\Large$\frac{z^2 - z + 2}{z^4 - 10z^2 + 9}$}$の極は$z =  i,3i$であり、\\
一位の極$z = i$における留数は、
\begin{align*}
Res\Bigl(\frac{z^2 - z + 2}{z^4 - 10z^2 + 9}, z = i\Bigr) &= \lim_{z \to i}\bigl(z - i\bigr)\frac{z^2 - z + 2}{z^4 - 10z^2 + 9}\\
&= \lim_{z \to i}\frac{z^2 - z + 2}{(z+i)(z+3i)(z-3i)}\\
&= \frac{-1-i+2}{2i\times 4i \times (-2i)}\\
&= \frac{1-i}{16i}\\
&= -\frac{1+i}{16}
\end{align*}
一位の極$z = 3i$における留数は、
\begin{align*}
Res\Bigl(\frac{z^2 - z + 2}{z^4 - 10z^2 + 9}, z = 3i\Bigr) &= \lim_{z \to 3i}\bigl(z - 3i\bigr)\frac{z^2 - z + 2}{z^4 - 10z^2 + 9}\\
&= \lim_{z \to 3i}\frac{z^2 - z + 2}{(z-i)(z+i)(z+3i)}\\
&= \frac{-9-3i+2}{2i\times 4i \times 6i}\\
&= \frac{-7-3i}{-48i}\\
&= \frac{3-7i}{48}
\end{align*}
よって、
\begin{equation}
\label{1}
\int_{C_x + C_R}\mathrm{d}z\frac{z^2 - z + 2}{z^4 + 10z^2 + 9} = 2\pi i \times \bigl(-\frac{1+i}{16}\bigr)  + 2\pi i \times \bigl(\frac{3-7i}{48}\bigr) = \frac{5\pi}{12}
\end{equation}
一方、
\begin{equation}
\label{2}
\int_{C_x + C_R}\mathrm{d}z\frac{z^2 - z + 2}{z^4 + 10z^2 + 9} = \int_{C_x}\mathrm{d}z\frac{z^2 - z + 2}{z^4 + 10z^2 + 9} + \int_{C_R}\mathrm{d}z\frac{z^2 - z + 2}{z^4 + 10z^2 + 9} = I
\end{equation}
(\ref{1})、(\ref{2})より、
$$
\therefore I = \frac{5\pi}{12}
$$
\\

%(3)
\subsection{}
$$
I = \int^{\infty}_{-\infty}\mathrm{d}x\frac{x^4}{(x^2+2)^2(x^2+3)}
$$
とおく。
$$
\int_{C_x}\mathrm{d}z\frac{z^4}{(z^2+2)^2(z^2+3)} = \int^{R}_{-R}\mathrm{d}x\frac{x^4}{(x^2+2)^2(x^2+3)} \to I \quad(R \to \infty)
$$
$z = Re^{i\theta}$とおくと、
\begin{align*}
\biggl|\int_{C_R}\mathrm{d}z\frac{z^4}{(z^2+2)^2(z^2+3)}\biggr| 
&= \biggl|\int^{\pi}_{0}i Re^{i\theta}\mathrm{d}\theta\frac{R^4e^{4i\theta}}{(R^2e^{2i\theta} + 2)^2(R^2e^{2i\theta} + 3)}\biggr|\\
&= \int^{\pi}_{0}\mathrm{d}\theta\Bigl|\frac{R^5e^{5i\theta}}{(R^2e^{2i\theta} + 2)^2(R^2e^{2i\theta} + 3)}\Bigr| \to 0 \quad(R \to \infty)\\
\therefore \quad\int_{C_R}\mathrm{d}z\frac{z^4}{(z^2+2)^2(z^2+3)} &\to 0 \quad(R \to \infty)
\end{align*}
経路$C_x + C_R$の内側にある$\mbox{\Large$\frac{z^4}{(z^2+2)^2(z^2+3)}$}$の極は$z =  \sqrt{2}i,\sqrt{3}i$であり、\\
二位の極$z = \sqrt{2}i$における留数は、
\begin{align*}
Res\Bigl(\frac{z^4}{(z^2+2)^2(z^2+3)}, z = \sqrt{2}i\Bigr) &= \lim_{z \to\sqrt{2}i}\frac{\mathrm{d}}{\mathrm{d}z}\biggl(\bigl(z - \sqrt{2}i\bigr)^2\frac{z^4}{(z^2+2)^2(z^2+3)}\biggr)\\
&= \lim_{z \to\sqrt{2}i}\frac{\mathrm{d}}{\mathrm{d}z}\biggl(\frac{z^4}{(z + \sqrt{2}i)^2(z^2+3)}\biggr)\\
&=  \lim_{z \to\sqrt{2}i}\frac{z^3(z + \sqrt{2}i)^2(z^2+3)-z^42(z + \sqrt{2}i)(z^2+3)-z^4(z + \sqrt{2}i)^22z}{(z + \sqrt{2}i)^4(z^2+3)^2}\\
&= \frac{-2\sqrt{2}i\times(2\sqrt{2}i)^2 \times (-2 + 3) - 16\times2\sqrt{2}i \times (-2 + 3) - 8 \times(2\sqrt{2}i)^2 \times2\sqrt{2}i}{(2\sqrt{2}i)^4 \times (-2 + 3)^2}\\
&= \frac{7\sqrt{2}}{4}i
\end{align*}
一位の極$z = \sqrt{3}i$における留数は、
\begin{align*}
Res\Bigl(\frac{z^4}{(z^2+2)^2(z^2+3)}, z = \sqrt{3}i\Bigr) &= \lim_{z \to \sqrt{3}i}\bigl(z - \sqrt{3}i\bigr)\frac{z^4}{(z^2+2)^2(z^2+3)}\\
&= \lim_{z \to \sqrt{3}i}\frac{z^4}{(z^2+2)^2(z+\sqrt{3}i)}\\
&= \frac{9}{(-3 + 2)^2 \times 2\sqrt{3}i}\\
&= -\frac{3\sqrt{3}i}{2}
\end{align*}
よって、
\begin{equation}
\label{5}
\int_{C_x + C_R}\mathrm{d}z\frac{z^4}{(z^2+2)^2(z^2+3)} = 2\pi i \times \bigl(\frac{7\sqrt{2}}{4}i\bigr)  + 2\pi i \times \bigl(-\frac{3\sqrt{3}i}{2}\bigr) = \Bigl(3\sqrt{3} - \frac{7\sqrt{2}}{2}\Bigr)\pi
\end{equation}
一方、
\begin{equation}
\label{6}
\int_{C_x + C_R}\mathrm{d}z\frac{z^4}{(z^2+2)^2(z^2+3)} = \int_{C_x}\mathrm{d}z\frac{z^2 - z + 2}{z^4 + 10z^2 + 9} + \int_{C_R}\mathrm{d}z\frac{z^2 - z + 2}{z^4 + 10z^2 + 9} = I
\end{equation}
(\ref{5})、(\ref{6})より、
$$
\therefore I = \Bigl(3\sqrt{3} - \frac{7\sqrt{2}}{2}\Bigr)\pi
$$

\end{document}